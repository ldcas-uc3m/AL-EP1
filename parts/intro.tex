\section{Introducción}
Vivimos en un mundo completamente globalizado. La tecnología evoluciona a pasos agigantados y los organismos competentes que hacen cumplir la ley se ven en la necesidad de evolucionar rápidamente en consecuencia. Nos precede una recesión económica derivada de la COVID-19 y de un confinamiento que propulsó la digitalización y exaltó la necesidad del mundo de estar comunicado.

En este contexto, surge la necesidad en los pequeños comercios de digitalizar su negocio, obtener visibilidad y buscar alternativas que le ayuden a salir del pozo económico en el que se encuentran. Sin embargo, multitud de empresas tecnológicas se aprovechan de la situación con objetivos claramente económicos ajenos a toda moral o ética.

A continuación, mostraremos un ejemplo erróneo de tratamiento de datos, concretamente una empresa de mudanzas, \textit{Mudanzas Benji} \cite{benji}, tomando como base la legislación vigente: el Reglamento (UE) 2016/679 \cite{RGPD} (en adelante ``RGPD'') y la Ley Orgánica 3/2018 \cite{LOPDPGDD} (en adelante ``LOPDPGDD'').

Posteriormente, expondremos qué presuntos incumplimientos ofrece y qué consecuencias legales acarrean estas acciones. Finalmente, visualizaremos otra empresa de mudanzas, \textit{Mudanzas Segoviana Madrid} \cite{alcorcon}, que cumple adecuadamente con los estándares de privacidad.
