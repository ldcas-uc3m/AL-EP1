\section{Consecuencias}
% Sanciones:
% - RGPD art. 82.1 y 83.5 y 84.1
% - LOPDPGDD título IX
%   - 72.1.b/c/k
%   - 74.a/e/i, 76.1
%   - 78

Según el artículo 82.1 del RGPD, \textquote{[t]oda persona que haya sufrido daños y perjuicios materiales o inmateriales como consecuencia de una infracción del presente Reglamento tendrá derecho a recibir del responsable o el encargado del tratamiento una indemnización por los daños y perjuicios sufridos}.

El artículo 84.1 del mismo dice que \textquote{[l]os Estados miembros establecerán las normas en materia de otras sanciones aplicables a las infracciones del presente Reglamento}, por lo que es necesario consultar la LOPDPGDD.

\subsection{Presuntas infracciones}
Según el artículo 72.1, apartados b y c, de la LOPDPGDD, se considera una infracción \textquote{muy grave} \textquote{[e]l tratamiento de datos personales sin que concurra alguna de las condiciones de licitud del tratamiento establecidas en el artículo 6 del Reglamento (UE) 2016/679}, y \textquote{El incumplimiento de los requisitos exigidos por el artículo 7 del Reglamento (UE) 2016/679 para la validez del consentimiento}. Como mencionamos anteriormente en las Subsecciones \ref{subsec:contacto} y \ref{subsec:cookies}, el formulario de contacto y las \textit{cookies} incumplen, presunta y parcialmente, ambos artículos, por lo que \textit{Mudanzas Benji} incurre, presuntamente, en infracciones muy graves.

Según el artículo 74.a, se consideran infracciones \textquote{leves} \textquote{[e]l incumplimiento del principio de transparencia de la información o el derecho de información del afectado por no facilitar toda la información exigida por los artículos 13 y 14 del Reglamento (UE) 2016/679}. Como mencionamos anteriormente en la \subsecref{privacidad}, la política de privacidad incumple, presunta y parcialmente, el artículo 13, por lo que \textit{Mudanzas Benji} incurre, presuntamente, en infracciones leves.



\subsection{Sanciones}
Según el artículo 83.5 del RGPD, \textquote{[l]as infracciones de las disposiciones siguientes se sancionarán, de acuerdo con el apartado 2, con multas administrativas de 20 000 000 EUR como máximo o, tratándose de una empresa, de una cuantía equivalente al 4 \% como máximo del volumen de negocio total anual global del ejercicio financiero anterior, optándose por la de mayor cuantía:} \textquote{los principios básicos para el tratamiento, incluidas las condiciones para el consentimiento a tenor de los artículos 5, 6, 7 y 9} (apartado a) y \textquote{los derechos de los interesados a tenor de los artículos 12 a 22} (apartado b).

Según el artículo 83.1, \textquote{Cada autoridad de control garantizará que la imposición de las multas administrativas con arreglo al presente artículo por las infracciones del presente Reglamento indicadas en los apartados 4, 5 y 6 sean en cada caso individual efectivas, proporcionadas y disuasorias}; y según el 83.2, \textquote{[l]as multas administrativas se impondrán, en función de las circunstancias de cada caso individual}.

Dado que factores como la gravedad o la negligencia son los que deben tenerse en cuenta para la aplicación de éstas sanciones, y considerando las presuntas infracciones, no creemos que sean de suficiente gravedad como para suponer una sanción grave.