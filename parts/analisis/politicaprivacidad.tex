% Información a presentar:
% - RGPD art. 12.1, 12.3, 12.7, 13.1, 13.2
% - LOPDPGDD art. 11.2
\subsection{La política de privacidad}\label{subsec:privacidad}
En esta sección repasaremos la política de privacidad de \textit{Mundanzas Benji}\footurl{https://www.mudanzasbenji.com/politica-de-privacidad} y mencionaremos qué aspectos debe incluir y determinar si es completa.

\subsubsection{¿Dónde la podemos encontrar?}
Requiere paciencia encontrar en la parte inferior de la página la política de privacidad de \textit{Mudanzas Benji}, puesto que, recordemos, no está presente en el consentimiento (véase 2. El consentimiento).

El artículo 12.7 dictamina que \textquote{La información que deberá facilitarse a los interesados en virtud de los artículos 13 y 14 podrá transmitirse en combinación con iconos normalizados que permitan proporcionar de forma fácilmente visible, inteligible y claramente legible una adecuada visión de conjunto del tratamiento previsto}. 

En este caso, como podemos ver en la \figref{ubicacion_politica}, los enlaces a las políticas de privacidad y de \textit{cookies} aparecen en \textit{footer} de la página, bajo la sección \textquote{ENLACES LEGALES}. Éste lugar no es uno de los más visibles de toda la página, pero es claramente legible y de un tamaño similar al resto de la página.
También es cierto que está localizada en un sitio similar a donde aparece en la mayoría de las páginas web, y donde un usuario medio iría a buscarla.

\rasterfigure[.8]{ubicacion_politica}{Ubicación de la política de privacidad.}


\subsubsection{¿Qué lenguaje presenta?}
El vocabulario utilizado durante la exposición de la política de privacidad, o mejor dicho, el intento de reforzar y exponer a qué destinatarios van dirigidos la información de contacto y \textit{cookies}, es técnico y enrevesado. El uso de palabras complejas como \textquote{supresión}, \textquote{regir}, \textquote{competente} o \textquote{controversia} difieren significativamente con el objetivo fijado por el artículo 12.1 \textquote{[e]l responsable del tratamiento tomará las medidas oportunas para facilitar al interesado toda información indicada en los artículos 13 y 14, así como cualquier comunicación con arreglo a los artículos 15 a 22 y 34 relativa al tratamiento, en forma concisa, transparente, inteligible y de fácil acceso, con un lenguaje claro y sencillo, en particular cualquier información dirigida específicamente a un niño}. En nuestro caso, nos concierne el artículo 13, el cual menciona la información que deberá facilitarse cuando los datos personales se obtengan del interesado.


\subsubsection{Calidad del contenido}
A pesar de seguir un patrón coherente mencionando quién es el responsable del tratamiento, cuál es la necesidad del tratamiento y su finalidad, hay aspectos que quedan en el aire y que los artículos 13 y 14 del RGPD exigen, como por ejemplo, la ausencia del contacto del responsable de protección de datos dentro de la empresa. Según el artículo 13.1.b del RGPD, se ha de facilitar \textquote{los datos de contacto del delegado de protección de datos, en su caso}.

Otro caso que genera controversia es el supuesto derecho que se reservan, citado textualmente, \textquote{a modificar su Política de Privacidad, de acuerdo a nuestro propio criterio}. Sin embargo, el artículo 23 del RGPD muestra rotundamente la posibilidad de que \textquote{[e]l Derecho de la Unión o de los Estados miembros que se aplique al responsable o el encargado del tratamiento limitar, a través de medidas legislativas, el alcance de las obligaciones y de los derechos establecidos en los artículos 12 [\dots]}, y no menciona en ningún punto la posibilidad de dejar al libre albedrío estas decisiones.

\noindent
Por lo que respecta a la información restante, presenta una estructura clara y concisa.
