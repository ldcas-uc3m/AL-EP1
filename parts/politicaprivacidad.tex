% Información a presentar:
% - RGPD art. 12.1, 12.3, 12.7, 13.1, 13.2
% - LOPDPGDD art. 11.2
\section{La política de privacidad}
En esta sección repasaremos la política de privacidad de \textit{Mundanzas Benji} \cite{benjiprivacidad} y mencionaremos qué aspectos debe incluir y determinar si es completa.

\subsection{¿Dónde la podemos encontrar?}
Requiere paciencia encontrar en la parte inferior de la página la política de privacidad de \textit{Mudanzas Benji}, puesto que, recordemos, no está presente en el consentimiento (véase 2. El consentimiento).

El artículo 12.7 dictamina que \textquote{La información que deberá facilitarse a los interesados en virtud de los artículos 13 y 14 podrá transmitirse en combinación con iconos normalizados que permitan proporcionar de forma fácilmente visible, inteligible y claramente legible una adecuada visión de conjunto del tratamiento previsto}. 

En este caso, como podemos ver en la \figref{ubicacion_politica}, la visibilidad de la política se encuentra comprometida.
\rasterfigure[.7]{ubicacion_politica}{Ubicación de la política de privacidad.}

\newpage
\subsection{¿Qué lenguaje presenta?}
El vocabulario utilizado durante la exposición de la política de privacidad, o mejor dicho, intento de reforzar y exponer a qué destinatarios van dirigidos la información de contacto y \textit{cookies}, es técnico y enrevesado. El uso de palabras complejas como supresión, regir, competente o controversia difieren significativamente con el objetivo fijado por el artículo 12.1 \textquote{El responsable del tratamiento tomará las medidas oportunas para facilitar al interesado toda información indicada en los artículos 13 y 14, así como cualquier comunicación con arreglo a los artículos 15 a 22 y 34 relativa al tratamiento, en forma concisa, transparente, inteligible y de fácil acceso, con un lenguaje claro y sencillo, en particular cualquier información dirigida específicamente a un niño}. En nuestro caso actual, nos concierne el artículo 13 que menciona la información que deberá facilitarse cuando los datos personales se obtengan del interesado.

\subsection{Calidad del contenido}
A pesar de seguir un patrón coherente mencionando quién es el responsable del tratamiento, cuál es la necesidad del tratamiento y su finalidad, hay aspectos que quedan en el aire y que los artículos 13 y 14 del RGPD exigen, como por ejemplo, la ausencia del contacto del responsable de protección de datos dentro de la empresa [Artículo 13.1.b].

Otro caso que genera controversia es el supuesto derecho que se reservan, citado textualmente, \textquote{Nos reservamos el derecho a modificar su Política de Privacidad, de acuerdo a nuestro propio criterio}. Sin embargo, el artículo 23 del RGPD muestra rotundamente la posibilidad de que el encargado de los datos pueda \textbf{LIMITAR} a través de medidas legislativas, el alcance de las obligaciones y derechos establecidos en el Artículo 12, y no menciona en ningún momento la posibilidad de dejar al libre albedrío estas decisiones.

Por lo que respecta a la información restante, presenta una estructura clara y concisa.
