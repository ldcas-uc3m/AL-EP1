\section{Presuntos incumplimientos}
En esta sección discutiremos los presuntos incumplimientos que \textit{Mudanzas Benji} infringe relacionados con el consentimiento y la información a presentar relativa a la política de privacidad \footurl{https://www.mudanzasbenji.com/politica-de-privacidad}.

Estamos amparados por el artículo 77.1 del RGPD que dicta que \textquote{todo interesado tendrá derecho a presentar
una reclamación ante una autoridad de control [\dots], si considera que el tratamiento de datos personales que le conciernen infringe el presente Reglamento}.


\subsection{El consentimiento}
% RGPD art. 4.1, 4.2, 4.11, 6.1.a, 7.1
% LOPDPGDD art. 6.1
En esta sección nos centraremos en el formulario de contacto\footurl{https://www.mudanzasbenji.com/contacto1} y uso de \textit{cookies} de \textit{Mudanzas Benji}.

\newpage
\subsubsection{El formulario de contacto}
El artículo 4.1 del RGPD define los datos personales como \textquote{toda información sobre una persona física identificada o identificable}, y especifica que \textquote{[u]na persona se considerará persona física identificable, aquella reconocible a través de un identificador}. Enlazando con el posterior artículo 4.2, el cual define el tratamiento de datos como \textquote{cualquier operación o conjunto de operaciones realizadas sobre datos personales o conjuntos de datos personales}, y con el artículo 4.11, el cual define el consentimiento como \textquote{toda manifestación de voluntad libre, específica, informada e inequívoca por la que el interesado acepta, ya sea mediante una declaración o una clara acción afirmativa, el tratamiento de datos personales que le conciernen}, podemos extraer que es necesario dar un consentimiento afirmativo para que se puedan tratar los datos personales. Este artículo entra en juego en cualquier formulario de contacto digital o a papel, que requiera de la necesidad de obtener información sensible del usuario.

En este contexto, el artículo 6.1.a) del RGPD refuerza la licitud del tratamiento de los datos adherido a un consentimiento, que choca con el formulario presentado por \textit{Mudanzas Benji}, ya que el tratamiento de los datos sólo será lícito si \textquote{el interesado dio su consentimiento para el tratamiento de sus datos personales}. La \figref{formulario_contacto_mudanzasbenji} incorpora la prueba de ausencia de afirmación por parte del interesado, atentando absolutamente contra el artículo 6.1 de la LOPDPGDD por el cual \textquote{se entiende por consentimiento del afectado toda manifestación de voluntad libre, específica, informada e inequívoca por la que este acepta, ya sea mediante una declaración o una clara acción afirmativa, el tratamiento de datos personales que le conciernen}.

\rasterfigure[.85]{formulario_contacto_mudanzasbenji}{Formulario de contacto de Mudanzas Benji.}

También cabe mencionar el artículo 7.1 del RGPD, que obliga a que \textquote{[c]uando el tratamiento se base en el consentimiento del interesado, el responsable deberá ser capaz de demostrar que aquel consintió el tratamiento de sus datos personales}.



\subsubsection{Las \textit{cookies} como robo de información}
Otra prueba irrefutable de una clara infracción a la LOPDPGDD y en consecuencia al RGPD, es el robo de información por medio de cookies. Es de obligado cumplimiento informar al usuario de la recogida de información delicada por medio de \textit{cookies} y de la disponibilidad de una política de cookies.

Inspeccionando el tráfico de red desde la consola del desarrollador de Google Chrome, hemos podido observar envío de información al servidor en formato de \textit{cookie} de Adobe Cloud\footnote{Específicamente, la \textit{cookie} \texttt{AMCV\_802D40325A94210E0A495D7F\%40AdobeOrg}} (véase \figref{cookies_adobe}).

\rasterfigure[.7]{cookies_adobe}{\textit{Cookie} interceptada desde el navegador.}

Indagando por la red para obtener información acerca de la \textit{cookie} de la \figref{cookies_adobe}, hemos recogido ciertas características interesantes:
\begin{itemize}
    \item \textbf{Duración:} 2 años.
    \item \textbf{Descripción:} Este es un nombre de \textit{cookie} de tipo patrón asociado con Adobe Marketing Cloud. Almacena un identificador de visitante único y utiliza un identificador de organización para permitir a una empresa realizar un seguimiento de los usuarios en todos sus dominios y servicios.
    \item \textbf{Tipo:} Terceros.
    \item \textbf{Finalidad:} Técnica.
\end{itemize}

Desafortunadamente, no es posible obtener de forma directa qué información recogen del usuario. Sin embargo, utilizando \textit{Cover your tracks} \cite{coveryourtracks}, podemos visualizar qué superficie de ataque ofrece nuestro usuario a la red y declarar qué metainformación expone potencialmente y qué datos han podido ser extraídos por la \textit{cookie}.

% Información a presentar:
% - RGPD art. 12.1, 12.3, 12.7, 13.1, 13.2
% - LOPDPGDD art. 11.2
% Derecho a denunciar:
% - RGPD art. 77
% Extras:
% - LOPDPGDD art. 2.1, 28